
\documentclass{kyvernitis-resume}
\fullname{Noah Schlenker}
% \jobtitle{Sofware Engineer}
\usepackage{hyperref}

\usepackage{eso-pic}

\begin{document}
\resumeheader
{\linkedin{noah-schlenker}}
{\email{noah.schlenker2002@gmail.com}}
{\github{noahpy}}
{}{}{}


\begin{section}{Education}
    \begin{subsectionnolist}{M.Sc. Mathmatics in Data Science}{Admitted}{Oct 2025 -- Oct 2027}{Technical University of Munich}
    \end{subsectionnolist}
    \begin{subsectionnolist}{B.Sc. Informatics}{GPA: 3.6 (US), 1.6 (GER) | Minor: Mathematics}{Oct 2021 -- May 2025}{Technical University of Munich}
    \end{subsectionnolist}

    
    \begin{subsectionnolist}{High school diploma (Abitur)}{GPA: 3.9 (US), 1.1 (GER)}{Sep 2013 -- Jul 2021}{IKG Leinfelden}
        % \italicitem{Major: Physics (15/15), Mathematics (14/15), English (13/15)}
        % \italicitem{GPA: 3.9 (US), 1.1 (GER)}
    \end{subsectionnolist}
\end{section}

\begin{section}{Work / Research Experience}

    \begin{subsection}{Bachelor's thesis}{Chair of Scientific Computing, TUM | Grade 100\%}{Oct 2024 -- Feb 2025}{Munich, GER}
    % \item[Grade: ]{}
    % \item{Research on parallelizing molecular dynamics simulations with 3-body interactions using \href{https://www.sciencedirect.com/science/article/abs/pii/S0021999106002919}{Zonal Methods}}
    \item{Research on node-level parallelization of high-fidelity molecular dynamics simulations}
    % \item{Derived theorectical requirements for applying Zonal Methods to 3-body interactions}

    \item{Efficient framework on multi-node simulator, demonstrating novel parallelization methods} 
        %using \href{https://github.com/AutoPas/AutoPas}{AutoPas}.}
\end{subsection}
    \textit{Tech Stack}: C++, C, Slurm linux cluster, Paraview, MPI\hspace{0.1cm} | \hspace{0.1cm}
    \textit{Links}: \href{https://mediatum.ub.tum.de/1781358}{Thesis}
    

    \begin{subsection}{Research Intern at "AI Factory Bavaria"}{Chair of Robotics, Artificial Intelligence and Real-Time Systems, TUM}{Feb 2024 -- Aug 2024}{Munich, GER}
    \item Research on task and motion planning of mobile base + manipulator robots
    \item ROS-package for combined motion planning yielding up to double motion efficiency
    \end{subsection}
    \textit{Tech Stack}: ROS, C++, MoveIt, Rviz, Ubuntu\hspace{0.1cm} | \hspace{0.1cm}
    \textit{Links}: \href{https://kifabrik.mirmi.tum.de/}{AI Factory Bavaria}
    

    \begin{subsection}{Software Engineer Working Student}{Entrix Energy}{Dec 2022 -- Jul 2023}{Munich, GER}
        \item Architecting optimized algo-testing service using historic energy market behaviour
        \item Facilitated revenue forecasting and optimized development of the trading algorithm
        % \item Achieved tenfold reduction in runtime compared to the previous service
    \end{subsection}
    \textit{Tech Stack}: AWS, Python, Typescript \hspace{0.1cm} | \hspace{0.1cm}
    \textit{Links}: \href{https://www.entrixenergy.com/en/home/}{Entrix Energy}
    
    
    % \begin{subsection}{Freelance project}{Co-Founder / Software Engineer}{Mar 2022 -- Aug 2022}{Munich, GER}
    %     \item Leading a team of 3 to develop a customer advisory application for a local artisan business
    %     \item Introduced transparency and calculability of projects for third customers
    %     \item Accelerated and visualized the project planning process by half
    % \end{subsection}
    % \textit{Tech Stack}: Flutter, Dart, Windows
    % 

  
\end{section}
\begin{section}{Extracurricular Activities}
     
    \begin{subsectionnosl}{Project: Quantitive Trading using ML models}{Apr 2025 - Present}
     \item{Fetch data from Bybit API, retrieve features, train models and backtest / live mock trade strategies}
     \item{Trained transformer model on predicting midprice with 75\% trend accuracy}
     % \item{Maintaining network infrastructure and internet access for the inhabitants}
     \end{subsectionnosl}
    \textit{Tech Stack}: Pytorch, Numpy, Bybit\hspace{0.1cm} | \hspace{0.1cm}
    \textit{Links}: \href{https://github.com/noahpy/my_ml_crypto_trading}{GitHub}

    \begin{subsectionnosl}{Project: Pytorch toolkit}{Jul 2025 - Present}
     \item{Python package for seamless training, evaluation and visualization using Pytorch}
     \item{Significant code reduction and efficient hyperparameter tuning}
     \end{subsectionnosl}
    \textit{Tech Stack}: Pytorch, Numpy \hspace{0.1cm} | \hspace{0.1cm}
    \textit{Links}: \href{https://github.com/noahpy/my_pytorch_kit}{GitHub}

     \begin{subsectionnosl}{Network / System Administrator}{Apr 2024 - Present}
     \item{Implementing / maintaining services within dormitory: dorm management sofware, cloud storage, WPAs}
     % \item{Maintaining network infrastructure and internet access for the inhabitants}
     \end{subsectionnosl}
    % \textit{Tech Stack}: Proxmox, VLAN, Pfsense, Wireguard, Debian\hspace{0.1cm} | \hspace{0.1cm}
    % \textit{Links}: \href{https://www.schollheim.net/}{Dormitory}
    


        % \pagebreak
        % \pagebreak

        %  \begin{subsection}{Leader and Organizer}{30th Kobe International Students' Symposium}{Dec 2024}{Osaka, JPN}
        %  \item{Planned / implemented bilingual (JP and EN) symposium aimed at at developing leadership and intercultural competency, hosted by Kobe University}
        %  % \item{Held a weekly 90-minute preparation course of the symposium, consisting of 30 participants from 16 countries to debate and collaborate on its organization}
        % \end{subsection}
        % \textit{Links}: \href{https://www-kobe--u-ac-jp.translate.goog/ja/announcement/20250106-66392/?_x_tr_sl=auto&_x_tr_tl=en&_x_tr_hl=en&_x_tr_pto=wapp}{KISS (Translated)}, \href{https://kym22-web.ofc.kobe-u.ac.jp/kobe_e_syllabus/2024/20/data/2024_3U084.html}{Course description}
    

     % \begin{subsection}{Advent of Code}{Yearly participation of daily puzzle solving challange}{2021-2024}{}
     % \item[]{}
     % \end{subsection}


    %  \begin{subsection}{Hackatum 2023}{Invited to discuss solution at Check24 HQ}{Jun 2023}{Munich, GER}
    %     \item{Developed a performance enhanced distance and rating based craftsman comparison algorithm}
    %     % \item{Core idea: Enable BFS through geological craftsman data composed as neighbouring networks, optimized by a C++ / Redis backend}

    %     \end{subsection}
    %     \textit{Tech Stack}: C++, Redis, NextJS\hspace{0.1cm} | \hspace{0.1cm}
    %     \textit{Links}: \href{https://github.com/zhngharry/HackaTUM-2023-Submission}{GitHub}
        

    % \begin{subsection}{START Hack 2023}{Europe-wide event with 1500 participants}{Mar 2023}{St. Gallen, SWZ}
    %     \item{Developed an user-experience centered android application for extending the mordern use of broadcast TV (Sunrise Challenge)}
    %     \item{Core idea: Enhance user interaction with the platfrom and other users to enrich the user experience, whilst enabling personalized ads for increased revenue}

    %     \end{subsection}
    %     \textit{Tech Stack}: Go, SQLite, React Native \hspace{0.1cm} | \hspace{0.1cm}
    %     \textit{Links}: \href{https://www.youtube.com/watch?v=rgmPpHwMYXU}{Demo Video}, \href{https://github.com/4rneee/STARTHack23-sunrise-backend}{GitHub}
        

    % \begin{subsection}{Ferienakademie 2022}{Selected course participant}{Sep 2022}{Sarntal, ITL}
    %     \item{Developed a 3D PSO Vizualization for the course "Modern Approaches to Optimization and Verification"}
    %     \item{Achieved efficient visualization and high performance by segregating Python for visualization and C for computational tasks}

    %     \end{subsection}
    %     \textit{Tech Stack}: C, Python, Pandas 3D Engine \newline
    %     \textit{Links}: \href{https://github.com/noahpy/PSO-Visualization}{\underline{GitHub}}, \href{https://ferienakademie.de/en/home-2/}{\underline{Ferienakademie}}
    % 


    % \begin{subsection}{HackaTUM 2022}{Case 3rd place out of 193 total teams}{Jun 2022}{Munich, GER}
    %     \item{Developed and implemented a chat-based movie recommendation system using AI-based NLP in 36 hours}

    %     \end{subsection}
    %     \textit{Tech Stack}: NLP, Kotlin, Flutter, PostgreSQL \newline
    %     \textit{Links}: \href{https://devpost.com/software/wachat}{\underline{Devpost}}
    % 


    %  \begin{subsection}{German Competitive Programming Contest 2022}{Price-Winning participation out of 80 teams}{Jun 2022}{Munich, GER}
    %  \item{6 hour long germany-wide competitive programming competition with two other peers}
    %  \item{Ranking criteria: solve as many questions whilst passing a required runtime benchmark}
    %     \end{subsection}
    %  \textit{Tech Stack}: Java
    % 


    % \begin{subsection}{Voluntary childcare and tutoring}{Stadtjugendring L.E.}{Sep 2020 -- Jul 2021}{Leinfelden, GER}
    %     \item{Trained and certified youth guide}
    %     \item{Organized sporting events and tutoring sessions for middle school student at my high school}

    %    \end{subsection}

 

\end{section}

\sectiontable{Skills}{
    \entry{Programming Languages}{C++, Python, C, Java, Dart, Go, JS, TS, OCaml}
    \entry{Expertise and Tools}{Machine Learning, Optimization, AWS, Design Patterns, Probability Theory, Linear Algebra, DBMS, Numeric Programming, Energy Sector}
    \entry{Languages}{Fluent in German, Japanese and English}
}

% insert as footnote: "References available upon request"
\AddToShipoutPictureBG*{%
  \AtPageLowerLeft{%
    \put(440, 12){\makebox(0,0)[lt]{\footnotesize References available upon request}}%
  }%
}

\end{document}
